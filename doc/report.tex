\documentclass[a4paper,12pt]{article}
\usepackage[utf8]{inputenc}
\usepackage{graphicx}
\usepackage{geometry}
\usepackage{hyperref}
\usepackage{listings}
\usepackage{xcolor}
\usepackage{float}
\usepackage{booktabs}
\usepackage{array}

\geometry{
    a4paper,
    total={170mm,257mm},
    left=20mm,
    top=20mm,
}

\title{
    \textbf{Smart Tank Monitoring System} \\
    \large IoT Assignment \#03
}
\author{Elber}
\date{\today}

\begin{document}

\maketitle

\vspace{2cm}

\begin{abstract}
The Smart Tank Monitoring System is a modular IoT solution designed to monitor rainwater levels in a tank and control a water channel valve. The system operates in two main modes: AUTOMATIC (software-controlled) and MANUAL (operator-controlled). It integrates four subsystems: an ESP32-based monitoring unit (TMS), an Arduino-based actuator unit (WCS), a Central Control Unit (CUS), and a Web Dashboard (DBS), communicating via MQTT, Serial, and HTTP protocols.
\end{abstract}

\newpage
\tableofcontents
\newpage

\section{System Overview}
The system creates a distributed architecture for tank monitoring and control. It addresses the requirement of managing water levels to prevent overflow or dry states by controlling a discharge valve.

The core functionalities are:
\begin{itemize}
    \item \textbf{Monitoring}: Continuous reading of water levels using sonar.
    \item \textbf{Control}: Automated valve actuation based on defined logic policies or manual override.
    \item \textbf{Visualization}: Real-time dashboard for remote supervision.
    \item \textbf{Alerting}: Visual indicators (LEDs) and system states (Warning, Alarm).
\end{itemize}

\section{System Architecture}
The system is divided into four main subsystems:

\subsection{1. Tank Monitoring Subsystem (TMS)}
\textbf{Hardware}: ESP32 based microcontroller.\\
\textbf{Role}: Captures environmental data and handles network connectivity.\\
\textbf{Functionality}:
\begin{itemize}
    \item Measures water level using an HC-SR04 sonar sensor.
    \item Publishes data to the \texttt{tank/level} MQTT topic.
    \item Indicates connection status via Green (Connected) and Red (Error) LEDs.
\end{itemize}

\subsection{2. Water Channel Subsystem (WCS)}
\textbf{Hardware}: Arduino UNO.\\
\textbf{Role}: Physical actuation and local user interface.\\
\textbf{Functionality}:
\begin{itemize}
    \item Controls the servo motor for valve opening ($0^\circ$ to $180^\circ$ mapped to 0-100\%).
    \item Provides local manual control via a Potentiometer.
    \item Displays current status on an LCD screen.
    \item Communicates with the CUS via Serial (JSON protocol).
\end{itemize}

\subsection{3. Control Unit Subsystem (CUS)}
\textbf{Software}: Java Application running on a PC/Server.\\
\textbf{Role}: The brain of the system, implementing the control logic.\\
\textbf{Functionality}:
\begin{itemize}
    \item Acts as a bridge between MQTT (TMS), Serial (WCS), and HTTP (DBS).
    \item Implements the automatic control policy based on thresholds $L1, L2$ and times $T1, T2$.
    \item Manages the global system state.
\end{itemize}

\subsection{4. Dashboard Subsystem (DBS)}
\textbf{Software}: Web Application (HTML/CSS/JS).\\
\textbf{Role}: Remote monitoring and control interface.\\
\textbf{Functionality}:
\begin{itemize}
    \item Visualizes real-time water level graphs.
    \item Allows mode switching (Automatic $\leftrightarrow$ Manual).
    \item Provides remote manual control of the valve.
\end{itemize}

\newpage

\section{Control Logic and FSMs}

\subsection{Mode Management}
The system operates in two mutually exclusive modes:
\begin{itemize}
    \item \textbf{AUTOMATIC}: The CUS determines the valve opening based on sensor readings.
    \item \textbf{MANUAL}: The user controls the valve setting via the Potentiometer (local) or the Dashboard (remote).
\end{itemize}
A timeout mechanism ensures safety: if the TMS stops sending data for $T2$ seconds, the system enters an \textbf{UNCONNECTED} state.

\subsection{Automatic Control Policy}
The automatic logic uses two water level thresholds ($L_1 < L_2$) and a timing threshold ($T_1$).

\begin{table}[H]
    \centering
    \renewcommand{\arraystretch}{1.5}
    \begin{tabular}{|l|l|l|}
        \hline
        \textbf{Condition} & \textbf{Valve Action} & \textbf{Description} \\
        \hline
        $Level < L_1$ & \textbf{0\% (Closed)} & Normal operation. \\
        \hline
        $L_1 < Level < L_2$ & \textbf{50\% (Half)} & \textit{Warning state}. Valve opens after duration $\ge T_1$. \\
        \hline
        $Level \ge L_2$ & \textbf{100\% (Open)} & \textit{Alarm state}. Valve opens immediately. \\
        \hline
    \end{tabular}
    \caption{Automatic Control Logic Table}
    \label{tab:logic}
\end{table}

\subsection{Finite State Machines (FSM)}

\subsubsection{TMS FSM (ESP32)}
\begin{itemize}
    \item \texttt{STATE\_INITIALIZING}: Hardware setup.
    \item \texttt{STATE\_CONNECTING\_WIFI}: Connecting to WiFi network.
    \item \texttt{STATE\_CONNECTING\_MQTT}: Connecting to the MQTT broker.
    \item \texttt{STATE\_CONNECTED}: Normal operation, sensing and publishing.
    \item \texttt{STATE\_NETWORK\_ERROR}: Fallback state on connection failure.
\end{itemize}

\subsubsection{WCS FSM (Arduino)}
\begin{itemize}
    \item \texttt{MODE\_UNCONNECTED}: Safe state, waiting for CUS heartbeat.
    \item \texttt{MODE\_AUTOMATIC}: Actuator follows CUS commands.
    \item \texttt{MODE\_MANUAL}: Actuator follows Potentiometer/Dashboard input.
\end{itemize}

\section{Hardware Implementation}

\subsection{TMS Connections (ESP32)}
\begin{itemize}
    \item \textbf{Sonar Trigger}: GPIO 13
    \item \textbf{Sonar Echo}: GPIO 12
    \item \textbf{Green LED}: GPIO 2
    \item \textbf{Red LED}: GPIO 4
\end{itemize}

\subsection{WCS Connections (Arduino UNO)}
\begin{itemize}
    \item \textbf{Servo Motor}: Pin 9 (PWM)
    \item \textbf{Potentiometer}: Pin A0 (Analog Input)
    \item \textbf{Button}: Pin 2 (Digital Input with Interrupt)
    \item \textbf{LCD Display}: I2C Bus (SDA=A4, SCL=A5)
\end{itemize}

\end{document}
